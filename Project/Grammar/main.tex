\documentclass[12pt, letterpaper]{article}
\usepackage[utf8]{inputenc}
\usepackage{graphicx}
\graphicspath{ {images/} }
\usepackage[document]{ragged2e}
\usepackage{amsmath}
\usepackage{fancyhdr}
\usepackage{etoolbox}
\usepackage{lastpage}
\usepackage{anyfontsize}
\usepackage{mathptmx}
\usepackage[usenames, dvipsnames, table,xcdraw]{xcolor}
\usepackage{cite}
\usepackage[english]{babel}
\usepackage{amsmath}
\usepackage[colorinlistoftodos]{todonotes}
\usepackage{etoolbox}
\usepackage{indentfirst}
\usepackage{color}
\usepackage{verbatim}
\usepackage{pdfpages}
\usepackage{tikz}
\usetikzlibrary{automata,positioning, arrows, shapes.geometric}
\usepackage{circuitikz}
\usepackage{float}
\usepackage{amssymb}
\usepackage{fancyvrb}
\usepackage{sectsty}
\usepackage{lipsum}
\usepackage[explicit]{titlesec}
\usepackage[hidelinks, breaklinks=true]{hyperref}
\usepackage{enumitem}
\usetikzlibrary{decorations.text}





%Margins
\usepackage{geometry}
 \geometry{
 letterpaper,
 left=1in,
 top=1in,
 right=1in,
 bottom=1in,
 }
%


%Header + Footer
%\setlength{\headheight}{15pt} 
\pagestyle{fancy}
\fancyhf{}
\fancyhead[RE,LO]{\fontsize{10}{12}\selectfont CS 580}
\fancyhead[LE,RO]{\fontsize{10}{12}\selectfont Spring 2018}
%\fancyhead[CE,CO]{\fontsize{10}{12}\selectfont \rightmark}
\fancyfoot[CE,CO]{\fontsize{9}{11}\selectfont -\ \thepage\ of \pageref{LastPage}\ -}
%\fancyfoot[LE,RO]{\thepage}
 
\renewcommand{\headrulewidth}{1.0pt}
\renewcommand{\footrulewidth}{0.5pt}

%



%Code Format Config
\usepackage{listings}
\usepackage{color}

\definecolor{dkgreen}{rgb}{0,0.6,0}
\definecolor{gray}{rgb}{0.5,0.5,0.5}
\definecolor{mauve}{rgb}{0.58,0,0.82}

\lstset{
	  frame=none,
	  language=C,
	  aboveskip=3mm,
	  belowskip=3mm,
	  showstringspaces=false,
	  columns=flexible,
	  basicstyle={\small\ttfamily},
	  numbers=left,
	  numberstyle=\tiny\color{gray},
	  keywordstyle=\color{blue},
	  commentstyle=\color{dkgreen},
	  stringstyle=\color{mauve},
	  breaklines=true,
	  breakatwhitespace=true,
	  tabsize=3,
	  xleftmargin=0.5cm
}


%



%Table Caption
\usepackage{caption} 
\captionsetup[table]{skip=5pt}
%

%Colors
\definecolor{myblue}{RGB}{0, 0, 255}
\definecolor{mygreen}{RGB}{53, 65, 0}

%


%Custom Section for Problem Labels
\sectionfont{\normalsize}
%\titlespacing{command}{left spacing}{before spacing}{after spacing}[right]
% spacing: how to read {12pt plus 4pt minus 2pt}
%           12pt is what we would like the spacing to be
%           plus 4pt means that TeX can stretch it by at most 4pt
%           minus 2pt means that TeX can shrink it by at most 2pt
%       This is one example of the concept of, 'glue', in TeX
\titlespacing\section{0pt}{4pt plus 4pt minus 2pt}{4pt plus 2pt minus 2pt}[0pt]
\titleformat{\section}{\color{ForestGreen}\normalfont\large\bfseries}{}{0em}{#1\ \thesection:}


%


\begin{document}
\fontsize{12}{16}\selectfont

{\color{myblue} \textbf{Grammar:}}


\begin{lstlisting}[mathescape=true]
// List of Colons

semi_colon_list
	:  ';'  semi_colon_list_tail

semi_colon_list_tail
	: semi_colon_list
	| $\epsilon$
	
---

// Assignment Operators

assignment_operator //Code Gen = first. Add more later
	: '='
	| MUL_ASSIGN
	| DIV_ASSIGN
	| MOD_ASSIGN
	| ADD_ASSIGN
	| SUB_ASSIGN
	| LEFT_ASSIGN
	| RIGHT_ASSIGN
	| AND_ASSIGN
	| XOR_ASSIGN
	| OR_ASSIGN

---

//Data Types

typed_ID
	: type_specifier ID

type
	: CHAR
	| SHORT
	| INT
	| LONG
	| FLOAT
	| DOUBLE
	
type_specifier //TODO Const, Volatile later
	: type
	| VOID
	| SIGNED type
	| UNSIGNED type

---

//Type List

type_specifier_list
	:  type_specifier type_specifier_list_tail
	
type_specifier_list_tail
	:  $\epsilon$
	| ',' type_specifier_list
	
---	

//Parameter List

parameter_specifier_list
	:  type_specifier ID parameter_specifer_tail
	
parameter_specifer_list_tail
	: $\epsilon$
	| ','   parameter_specifier_list
	
---

//Program	
program
	:  body EOF

---

//Body

body_statement
	: statement body_statement_tail
	
body_statement_tail
	: $\epsilon$ 
	: body
	
body_direct_declaration
	: direct_declaration body_direct_declaration_tail
	
body_direct_declaration_tail
	: $\epsilon$ 
	: body
	
body_function_declaration
	: function_declaration body_function_declaration_tail
	
body_function_declaration_tail
	: $\epsilon$ 
	: body
	
body_function_prototype
	: function_declaration body_function_prototype_tail
	
body_function_prototype_tail
	: $\epsilon$ 
	: body

body
	: body_statement
	| body_direct_declaration
	| body_function_declaration
	| body_function_prototype

---

//Function Declaration

function_prefix
	:  typed_ID '('

function_prototype
	:  function_prefix type_specifier_list ')' semi_colon_list
	|  function_declaration_prefix semi_colon_list
	
function_declaration
	:  function_prefix parameter_specifier_list ')' '{' statement '}' 
	
function_declaration_tail
	|  semi_colon_list
	|  $\epsilon$ 

---

//Direct Declaration 

direct_declaration
	: type_ID semi_colon_list
	| SIGNED type ID semi_colon_list
	| UNSIGNED type ID semi_colon_list
	| VOID ID semi_colon_list 
	| CHAR ID '='  STRING_LITERAL semi_colon_list
	
---

//Statement and Statement List	
	
statement
	: compound_statement //Sub statements to loops, conditional, and code blocks
	| expression_statement //Assignment, Boolean, Arithmetic Expressions
	| selection_statement //IF Statements
	| iteration_statement	 //Loops
	| semi_colon_list // End of statement one or more ;
	| direct_declaration
			
statement_list
	: statement statement_list_tail

statement_list_tail
	: statment_list
	| $\epsilon$
	
	
compound_statement
	: '{' '}'
	| '{' statement_list '}'
	
	
	
expression_statement
	: ';'
	| expression ';'

//TODO Add Expression
expression	
	
selection_statement
	: IF '(' expression ')' statement
	| IF '(' expression ')' statement ELSE statement
	
	
	
iteration_statement
	: WHILE '(' expression ')' statement
	| FOR '(' expression_statement expression_statement ')' statement
	| FOR '(' expression_statement expression_statement expression ')' statement
	

	
	

\end{lstlisting}











\newpage


	\begin{equation} \label{eq_2_1}
		\begin{aligned}
			program &\Rightarrow body\ \  EOF
		\end{aligned}
	\end{equation}
	
	\hfill\\\hfill\\
	\begin{equation} \label{eq_2_1}
		\begin{aligned}
			body \Rightarrow\ \ &stmt \\
				&|\ \ stmt\ \ body
		\end{aligned}
	\end{equation}
	
	
	
	\hfill\\\hfill\\ 
	\begin{equation} \label{eq_2_1}
		\begin{aligned}
			stmt \Rightarrow\ \ &ID\ \ '=' \ \ expr \ \ ';'\\
				&|\ \ IF\ \ '('\ \ bexp \ \ ')' \ \ stmt\\
				&|\ \ WHILE\  \ '('\ \ bexp \ \ ')' \ \ stmt\\
				&| \ \  '\{' \ \ substmts\\
				&|\ \ ';'
		\end{aligned}
	\end{equation}
	
	
	
	
	\hfill\\\hfill\\
	\begin{equation} \label{eq_2_1}
		\begin{aligned}
			substmts \Rightarrow\ \ &stmt\ \ '\}'\\
				&|\ \ stmt\ \ substmts\\
				&|\ \  '\}'\\
		\end{aligned}
	\end{equation}
	
	
	\hfill\\\hfill\\
	\begin{equation} \label{eq_2_1}
		\begin{aligned}
			assignment \Rightarrow\ \ &ID\ \ '='\ \ expression\ \ ';'\\
		\end{aligned}
	\end{equation}
	
	
	\hfill\\\hfill\\
	\begin{equation} \label{eq_2_1}
		\begin{aligned}
			expression \Rightarrow\ \ &expr\\
				&|\ \ bexpr
		\end{aligned}
	\end{equation}
	
	
	
	\hfill\\\hfill\\
	Notes - Continue down this route
	
	TODO
	\begin{itemize}
		\item expr - arithmetic expressions. Make sure to get precedence (greater precedence last) and associativity correct.
		From \url{http://pages.cs.wisc.edu/~fischer/cs536.s08/course.hold/html/NOTES/3.CFG.html#assoc} Remove POW
		\begin{verbatim}
			exp      --> exp PLUS term        |  exp MINUS term      |  term
			term     --> term TIMES factor    |  term DIVIDE factor  |  factor
			factor   --> exponent POW factor  |  exponent
			exponent --> MINUS exponent       |  final
			final    --> INTLITERAL           |  LPAREN exp RPAREN

		
		\end{verbatim}
		
		
		\item bexpr - boolean expressions
		
		\begin{verbatim}
		bexp --> TRUE
		bexp --> FALSE
		bexp --> bexp OR bexp
		bexp --> bexp AND bexp
		bexp --> NOT bexp
		bexp --> LPAREN bexp RPAREN
		
		\end{verbatim}
		\item  stmt - add if and while loops to it. Need to make sure no ambiguity in control statements. 
	\end{itemize}


\end{document}

